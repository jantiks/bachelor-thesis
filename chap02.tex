\chapter{Structural Properties of Non-KM-forcing Graphs}

Earlier, in Theorem \ref{thm:3}, we saw a construction called \( Z(H) \), which generates graphs containing the necessary Kempe chains without introducing additional
edges that would force H to be a \( T \)-rooted minor ($T$ is the transversal of the coloring of $Z(H)$). This suggests that connectivity plays a role in determining whether a graph is KM-forcing.

By Theorem \ref{thm:5}, all graphs with five vertices and at most six edges are KM-forcing.
Our goal is to investigate the connectivity properties of graphs \( G \) in which some graph \( H \) is not a $T$-rooted minor of $G$ but appears as the routing graph \( H(G,\mathfrak{C}, T) \). We call such graphs $H$ as non-KM-forcing in $G$.
By the following lemma, which proves a stronger statement, we will show that if \( H \) has at least seven edges and five vertices, and there exists some \( G \) in which \( H \) is non-KM-forcing, then smallest such \( G \) is 2-connected.


\begin{lemma}
   \label{main-lemma}
 Let $H$ and $G$ be graphs, let $\mathfrak{C}$ be a coloring of $G$, and let $T$ be the corresponding transversal such that
   $H$ is isomorphic to $H(G, \mathfrak{C}, T)$.
   
 Suppose the following hold:
   \begin{enumerate}
      \item $H$ is not a $T$-rooted minor in $G$.
      \item Every proper subgraph $H'$ of $H$ is KM-forcing.
      \item If $H$ is non-KM-forcing in any graph $G' \ncong G$, then
     $$|V(G)| + |E(G)| \leq |V(G')| + |E(G')|.$$
   \end{enumerate}
 Then $G$ is 2-connected.
\end{lemma}
   
   \begin{proof}
 The condition (3) forces $G$ to be the minimal counter-example. By minimality of $G$, for each pair of distinct color classes $A, B \in \mathfrak{C}$, the subgraph of $G$ induced by $A \cup B$ is a single path connecting the corresponding transversal vertices in $T$.
   
 Assume, for contradiction, that $G$ is 1-connected. Then, there exists a cut vertex $x$
 splitting $G$ into two subgraphs $L$ and $R$ that intersect only at $x$ (see Figure~\ref{fig:general-case}).

    \begin{figure}[h]
      \centering
      \vspace{0.3cm}
      \includegraphics[width=6cm]{img/general-case.eps}
      \vspace{0.3cm}
      \caption{A cut vertex $x$ splitting $G$ into subgraphs $L$ and $R$.}
      \label{fig:general-case}
  \end{figure}
   
   \textbf{Case 1: $x$ is a transversal vertex.}
   
 In this case, every Kempe chain passing from $L$ to $R$ must pass through $x$.
 Since is a transversal, such a chain would connect $x$ to another transversal vertex,
 contradicting the minimal-path property above. Hence no transversal in $L$
 can connect to one in $R$ via a Kempe chain.
   
 Let $H_L$ be the subgraph of $H$ induced by the transversals in $L \cup \{x\}$,
 and let $H_R$ be defined similarly for $R$. By condition (2), 
 each is KM-forcing in the corresponding subgraph of $G$, 
 so there are rooted minors of $H_L$ in $L$ and $H_R$ in $R$. 
 Since these contractions occur on disjoint vertex sets (except $x$) and respect the transversal roots, 
 combining them results a rooted minor of $H$ in $G$, contradicting condition (1).
   
   \textbf{Case 2: $x$ is not a transversal vertex.}
   
 If no Kempe chain passes through $x$ between $L$ and $R$, the same argument as in Case 1 applies.
 Otherwise, without loss of generality, suppose there is a Kempe chain in color class $A$ passing
 through $x$ from some transversal in $L$ to one in $R$. Then $x$ also has color $A$.
   
 Let $H_L$ be the subgraph of $H$ induced by the transversals in $L$ together with the color-$A$ transversal.
 Construct $L'$ by replacing $x$ in $L$ with that transversal. Since all relevant Kempe chains remain,
 by condition (2), $H_L$ is KM-forcing in $L'$.
   
 Similarly, let $H_R$ be the subgraph of $H$ induced by the transversals in $R$ plus
 one transversal in $L$ connected by the $A$-Kempe chain. 
 Form $R'$ by adding a single edge from that transversal in $L$ to $R$ by creating an
 edge with $x$. This preserves the Kempe chains corresponding to edges of $H_R$. 
 Again, by condition (2), $H_R$ is KM-forcing in $R'$.
   
 Since the contractions realizing the minors of $H_L$ in $L'$ and of $H_R$ in $R'$ act on disjoint parts of $G$ (except the cut vertex), 
 they combine to give a rooted minor of $H$ in $G$, contradicting condition (1).
   
 In all cases, we reach a contradiction. Therefore, $G$ must be 2-connected.
   \end{proof}

   \begin{cor}
 For every graph $H$ on five vertices and at least seven edges, if there is a graph $G$ with coloring $\mathfrak{C}$
 and transversal $T$ such that $H$ is isomorphic to $H(G, \mathfrak{C}, T)$ but there is no $T$-rooted minor $H$ in $G$. Then
      $G$ is 2-connected
   \end{cor}
 
   \begin{proof}
 By Theorem \ref{thm:5}, all graphs with five vertices and, at most, six edges are KM-forcing.
 Hence, we can apply the Lemma \ref{main-lemma} and get that $G$ is 2-connected.
   \end{proof}
