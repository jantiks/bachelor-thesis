\chapter{Computer Enumeration for Finding Counter-Examples in \( K_6 \)}

Since it remains open whether \( K_5 \) and \( K_6 \) are \( km \)-forcing, it is natural to attempt to find a counter-example.
A single counter-example would show that the graph is not \( km \)-forcing.
We performed a computer enumeration of the process of generating graphs \( G \) with a coloring \( \mathfrak{C} \) and a transversal \( T \) for a given minor \( H \),
ensuring that \( H \) is always isomorphic to \( H(G, \mathfrak{C}, T) \). We then checked whether \( H \) appears as a rooted minor in \( G \).

The idea is similar to the previous chapter's construction of \( Z(G) \). 
We aim to maintain all necessary Kempe chains between transversal vertices while ensuring that these vertices have no more than the necessary degrees.
However, since Kriesell and Mohr \cite{matthias_2022} showed that for every graph \( G \) with at most six vertices, the construction \( Z(G) \) does not
provide a counter-example \ref{thm:zg-for-k_6} (i.e., \( G \) remains a rooted minor of \( Z(G) \)), for looking for counter-examples 
for $K_6$, we must consider graphs with at least thirteen vertices.

At a high level, our algorithm for finding possible counter-examples works as follows.
We begin with a given rooted minor \( H \) and construct a set of graphs with a predefined coloring but initially without edges.
The coloring is chosen so that the vertices of \( H \) are rooted in these graphs; in other words, the transversal set corresponds to the graph \( H \). 

Next, for each edge of $H$, we construct the Kempe chain for each graph \( G \) in this set. After the Kempe chains are built, 
we know that \( H \) is the routing graph of \( G \) given its coloring and transversal. 
Then, we check whether \( G \) contains \( H \) as a rooted minor. 
If we find a graph \( G \) where \( H \) is not a rooted minor, we have found a counter-example.

The algorithm follows these three main steps:
\begin{enumerate}
    \item Given a graph \( H \), generate a set of graphs \( G \), each with a coloring \( \mathfrak{C} \) and a transversal \( T \), such that \( T = H \).
    \item For each such empty graph \( G \) with its assigned coloring, construct the necessary Kempe chains.
    \item For each constructed graph \( G \), check whether \( G \) contains \( H \) as a rooted minor.
\end{enumerate}

\section{Step 1: Generating Candidate Graphs}
In the first step, we generate candidate graphs \(G\) that may be counter-examples, where each graph has \(n\) vertices.

The coloring of each graph $G$ should be such that we get graph $H$ as a transversal of its coloring. Therefore, $G$ will have the coloring of size $|V(H)|$.
We color the vertices of each graph \(G\) using \( |V(H)| \) colors in such a way that the colors are distributed as evenly as possible. 
The balanced distribution gives us more possibilities to create Kempe chains in the next step.

To achieve this, we divide the \(n\) vertices into layers, each complete layer containing exactly \( |V(H)| \) vertices. 
The vertices are assigned distinct colors in every complete layer, so each color appears exactly once.
The first layer will be the transversal of the coloring (layer 0).

Since \(m\) is not neccessarily an exact multiple of \( |V(H)| \), there can be a set of vertices that do not form a complete layer. 
For these vertices, we consider all possible ways of assigning the colors.
Each new coloring of these remaining vertices results in a different candidate graph.

\begin{example}
 Let \(H = K_3\) (so \( |V(H)| = 3 \)) and \(n = 8\). 
 In this case, we can form two complete layers of three vertices each, each layer colored with the three colors in a fixed order.
 The remaining two vertices can then be colored by choosing two available colors.
 There are \(\binom{3}{2}\) ways to select two distinct colors from three and three ways to assign the same color to both.
 Hence, this produces \(\binom{3}{2} + 3 = 6\) distinct candidate graphs.
\end{example}


\begin{algorithm}[H]
    \caption{\textsc{GenerateCandidateGraphs}$(H, n)$:}
    \label{alg:balanced-coloring}
    \begin{algorithmic}[1]
    \Require Graph $H$ all differently colored vertices, with $k = |V(H)|$, number of vertices $n$
    \Ensure Set of candidate graphs $\mathcal{G}$
    
    \State $L \gets \lfloor n / k \rfloor$ \Comment{Number of full layers}
    \State $R \gets n \bmod k$ \Comment{Number of leftover vertices}
    \State Initialize empty graph $G \gets \{\}$
    
    \For{$i \gets 0$ to $L-1$}
        \For{each vertex $v \in V(H)$}
            \State Create new vertex $v'$
            \State Set $v'.layer$ $\gets i$
            \State Set $v'.color$  $\gets v.color$
            \State Add $v'$ to $G$
        \EndFor
    \EndFor
    
    \State Initialize empty set of graphs $\mathcal{G} \gets \{\}$
    
    \If{$R = 0$}
        \State Add $G$ to $\mathcal{G}$ \Comment{We have only one generated graph}
    \Else
        \State $\mathcal{C} \gets$ all combinations with replacement of $R$ elements from $V(H)$
        \For{each combination $c \in \mathcal{C}$}
            \State $G' \gets$ copy of $G$
            \For{each $v \in c$}
                \State Create new vertex $v'$
                \State Set $v'.layer$ $\gets L$
                \State Set $v'.color$  $\gets v.color$
                \State Add $v'$ to $G$
            \EndFor
            \State Add $G'$ to $\mathcal{G}$
        \EndFor
    \EndIf
    
    \State \Return $\mathcal{G}$
    \end{algorithmic}
\end{algorithm}
    
\paragraph{Complexity Analysis.}
Let \( k = |V(H)| \) be the number of colors and \( n \) the number of vertices to generate.

\begin{itemize}
    \item In the beginning the algorithm creates \( L = \lfloor n / k \rfloor \) full layers, each with size \( k \). This takes \( \mathcal{O}(n) \) time.
    
    \item If \( R > 0 \), we generate all combinations with replacement of \( R \) elements from \( k \) choices. The number of such combinations is:
    \[
        \binom{k + R - 1}{R} = \mathcal{O}(k^R)
    \]
 For each combination, we create a new graph with \( R \) additional nodes, which takes \( \mathcal{O}(R) \) time for each graph. Hence, this step has time complexity of \( \mathcal{O}(R \cdot k^R) \).
    
    \item Overall, the total time complexity is $\mathcal{O}(n + R \cdot k^R)$

    \item Since $R \leq k$, in the worst case we get time complexity of $\mathcal{O}(n + k^{k+1})$
    
\end{itemize}


\section{Step 2: Constructing Kempe Chains}
Now, for each candidate graph $G$ and each edge of $H$, we construct a Kempe chain between the corresponding transversal vertices of $G$. After constructing all Kempe chains, 
we will have $H$ as the routing graph of $G$, its coloring, and the transversal. First, let us have a subroutine that checks if there is a Kempe chain
in the graph $G$ for two given vertices.

\begin{algorithm}[H]
    \caption{\textsc{ExistsKempe}$(G, \textit{current}, \textit{beginning}, \textit{end}, \textit{visited} = \emptyset)$}
    \label{alg:exists-kempe}
    \begin{algorithmic}[1]
    \Require graph $G$, current vertex, end vertex, beginning vertex, set of already visited vertices
    \Ensure \textbf{True} if there exists a Kempe chain between \textit{beginning} and \textit{end}, \textbf{False} otherwise
    
    \If{$\textit{current} = \textit{end}$}
        \State \Return \textbf{True}
    \EndIf
    
    \ForAll{$\textit{neighbor} \in current.neighbors$}
        \If{$neighbor.color = current.color$ or $\textit{neighbor} \in \textit{visited}$}
            \State \textbf{Skip iteration}
        \ElsIf{
            $neighbor.color$ is the alternating color of the beginning or end
 }
            \State \Return \textsc{ExistsKempe}$(G, \textit{neighbor}, \textit{beginning}, \textit{end}, \textit{visited}  \cup \{\textit{neighbor}\})$
        \EndIf
    \EndFor
    
    \State \Return \textbf{False}
    \end{algorithmic}
    \end{algorithm}

\paragraph{Complexity Analysis.}

This is a modified version of the DFS. Therefore, the time complexity is $O(m + n)$, where $m$ is the number of edges of the graph $G$, and 
$n$ is the number of vertices of the graph $G$.

Now, we can have the main algorithm, which is building a Kempe chain in a given graph $G$ with colored vertices for two given vertices. This function also 
adds new variations of graph $G$ into a common set in case there are multiple ways of building the Kempe chain between two vertices.

\begin{algorithm}[H]
    \caption{\textproc{BuildKempeChain}$(G, u, v_0, w_0, \mathcal{G}, C)$}
    \begin{algorithmic}[1]
        \Statex \textbf{Input:} 
        \begin{itemize}
            \item \( G \) — Graph with colored and layered vertices
            \item \( u \) — Current vertex
            \item \( v_0 \) — Start vertex
            \item \( w_0 \) — End vertex
            \item \( \mathcal{G} \) — Set of all generated graph variants
            \item \( C \) — the current Kempe chain
        \end{itemize}
        
        \Statex \textbf{Output:} 
        \begin{itemize}
            \item The graph \( G \), modified with an extended Kempe chain from \( v_0 \) to \( w_0 \) (If it exists).
            \item Additional graph variants, also containing the extended Kempe chain, added to the set \( \mathcal{G} \)
        \end{itemize}
    
    \If{$C$ is empty}
        \State $C \gets [u]$
    \EndIf
    \If{$u = w_0$}
        \State \Return \Comment{The Kempe chain exists}
    \EndIf
    
    \State Determine color $c$ to look for based on $v_0.\textcolor{gray}{color}$ and $w_0.\textcolor{gray}{color}$
    
    \State $availableVertices \gets$ all vertices with color $c$ and not in $C$.
    
    \If{$u.\textcolor{gray}{layer} = 0$}
        \State Filter $availableVertices$ to exclude vertices with $\textcolor{gray}{layer} = 0$
    \EndIf
        
    \If{$availableVertices = \emptyset$} 
        \If{$\textproc{ExistsKempe}(G, v_0, w_0, \{v_0\})$}
            \State \Return
        \EndIf
        \State Remove $G$ from $\mathcal{G}$ if present \Comment {Removing degenerate cases}
        \State \Return
    \EndIf


    \State \texttt{alreadyExtended} $\gets$ \textbf{false}
    
    \ForAll{$v \in availableVertices$}
        \If{\texttt{alreadyExtended}}
            \State $G' \gets$ copy of $G$
            \State Add edge $(u, v)$ to $G'$
            \State Add $G'$ to $\mathcal{G}$
            \State \Call{BuildKempeChain}{$G', v, v_0, w_0, \mathcal{G}, C \cup \{w\}$}
        \Else
            \State Add edge $(u, v)$ to $G$
            \State \Call{BuildKempeChain}{$G, v, v_0, w_0, \mathcal{G}, C \cup \{w\}$}
            \State \texttt{alreadyExtended} $\gets$ \textbf{true}
        \EndIf
    \EndFor
    
    \end{algorithmic}
    \end{algorithm}

\paragraph{Complexity Analysis.}


\section{Step 3: Checking for Rooted Minors}
Finally, for each constructed graph \( G \), we check if it contains \( H \) as a rooted minor. If any such \( G \) does not contain \( H \), we have successfully found a counter-example.
For this step, we use a heuristic tool for finding minor embeddings \cite{dwavesystems2023minorminer}. The main utility function that we used, \texttt{find\_embedding()},
is an implementation of the heuristic algorithm described in \cite{cai2014practicalheuristicfindinggraph}.

\begin{algorithm}[H]
    \caption{FindCounterExample($H$, $\mathcal{G}$)}
    \begin{algorithmic}[1]
        \Statex \textbf{Input:} 
        \begin{itemize}
            \item \( H \) — the target minor graph
            \item \( \mathcal{G} \) — a set of graphs to check
        \end{itemize}
        \Statex \textbf{Output:} A pair \( (G, H) \) where \( H \) is not a rooted minor of $G$.

        \For{$kempeGraph$ in $\mathcal{G}$}
            \State $embedding \gets find\_embedding(H, kempeGraph, suspend\_chains = H)$
            \If{$embedding = \emptyset$}
                \For{$i \gets 1$ \textbf{to} 10}
                    \State $embedding \gets find\_embedding(H, kempeGraph, random\_seed = randomNumber, suspend\_chains = H)$
                    \If{$embedding \neq \emptyset$}
                        \State \textbf{continue the outer loop} \Comment{Minor exists. Skip the iteration}
                    \EndIf
                \EndFor
                
                \State \Return  $(kempeGraph, H)$ 
            \EndIf
        \EndFor
    \end{algorithmic}
\end{algorithm}

\section{Bringing everything together}

The following algorithm formalizes this process:

\begin{algorithm}[H]
    \caption{Searches for counter examples}
    \label{alg:rooted_minor_search}
    \begin{algorithmic}[1]
            \Statex \textbf{Input:} 
            \begin{itemize}
                \item \( n \) — the number of vertices in the target minor \( H \)
                \item \( m \) — the number of vertices in the parent graph \( G \)
            \end{itemize}
            \Statex \textbf{Output:} A pair \( (G, H) \) where \( H \) is a routing graph of $G$ with it's corresponding coloring
 and transversal but not a rooted minor in \( G \), or \textbf{None} if no such pair exists.
            
            \State \( \textit{candidateMinors} \gets \Call{EnumerateNonKMForcingCandidates}{n} \)
            
            \For{\textbf{each} candidate minor \( H \in \textit{candidateMinors} \)}
                \State \( \textit{parentGraphs} \gets \Call{GenerateCandidateGraphs}{H, m} \)
                
                \For{\textbf{each} parent graph \( G \in \textit{parentGraphs} \)}
                    \State \( \textit{currentFilledGraphs} \gets [G] \)
                    
                    \For{\textbf{each} edge \( e \in E(H) \)}
                        \State \( \textit{kempeChain} \gets \Call{BuildKempeChain}{G, e, \textit{currentFilledGraphs}} \)
                    \EndFor
                    \State $result \gets$ \Call{FindCounterExample}{$H$, \textit{currentFilledGraphs}}
                    \If{$result \neq$ None}
                        \State \Return $result$ \Comment{Found a counter-example}
                    \EndIf
                \EndFor
            \EndFor
            
            \State \Return \textbf{None} \Comment{No counter-examples found}
    \end{algorithmic}
\end{algorithm}

