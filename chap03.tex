\chapter{Computer Enumeration for Finding Counter-Examples in \( K_6 \)}
\label{chap:enumeration}

Since it remains open whether \( K_5 \) and \( K_6 \) are KM-forcing, it is natural to attempt to find a counter-example.
A single counter-example would show that the graph is non-KM-forcing.

Using computer-assisted enumeration, we tried to find counter-examples for all spanning subgraphs $H$ of $K_6$ such that 
every vertex of $H$ has a degree of at least two, and at least one vertex of $H$ has a degree of at least three.
More precisely, for each such graph H, we enumerated all (up to isomorphism) triples $(G,\mathfrak{C}, T)$, where $G$ is
a graph with at most 13 vertices, $\mathfrak{C}$ is a coloring of $G$, and $T$ is a transversal such that $H(G,\mathfrak{C}, T)$ is isomorphic
to $H$. More precisely, we restricted the enumeration to the graphs $G$ that contain only the edges of the
Kempe paths corresponding to the edges of the routing graph $H(G,\mathfrak{C}, T)$; this is clearly without loss of generality.

We then checked whether \( H(G, \mathfrak{C}, T) \) is a $T$-rooted minor in $G$.

At a high level, our algorithm for finding possible counter-examples works as follows.
First, we fix the vertex set of $G$ and go over all
choices of the coloring $\mathfrak{C}$. Since we are generating the graphs up
to isomorphism, we can permute the vertices to any
 way we wish, and thus, it actually suffices to go over all possible
choices $n_1, \dots, n_k$ of the color class sizes and color the first $n_1$ vertices 
by color 1, the following $n_2$ vertices by color 2, etc. Moreover, we
can fix the transversal $T$ to contain exactly the first vertex in each color class.

Next, we construct graphs for each coloring where each edge of $H$ corresponds to a Kempe chain between the corresponding transversal vertices. After the Kempe chains are built, 
we know that \( H \) is isomorphic to the routing graph of all those graphs $G$ with their coloring $\mathfrak{C}$ and the transversal $T$. 
Then, we check whether those graphs contain $H(G, \mathfrak{C}, T)$ as a $T$-rooted minor. 

Let us now discuss each of the steps in detail.

\section{Step 1: Generating Colorings}
In the first step, we generate the colorings of the $n$ fixed vertices using exactly $k := |V(H)|$ colors.
Of these $n$ vertices, $k$ form the transversal $T$ and receive distinct colors. To the rest of~$n - k$ vertices, 
we assign all possible combinations with the replacement of the colors~$1, \dots k$. 
Thus, there are $\binom{n - 1}{k - 1}$ choices for the coloring $\mathfrak{C}$.
\begin{example}
 Let \(H = K_3\) (so \( |V(H)| = 3 \)) and let \(n = 8\). 
 First, we select three vertices and mark them as the transversal vertices.
 We give each of them a distinct color from $1, 2, 3$.
 Then, we have $\binom{7}{2} = 21$ ways to assign colors for the five remaining vertices. 
 Hence, we get 21 different colorings.
\end{example}


\begin{algorithm}[H]
    \caption{\textsc{GenerateColorings}$(H, n)$:}
    \label{alg:balanced-coloring}
    \begin{algorithmic}[1]
    \Require Graph $H$ all differently colored vertices, with $k = |V(H)|$, number of vertices $n$
    \Ensure Set of colorings $\mathcal{A}$

    \State Initialize empty coloring $\mathfrak{C} \gets \{\}$
    
    \For{each vertex $v \in V(H)$}
        \State Create new vertex $v'$
        \State Set $v'.isTransversal$ $\gets True$
        \State Set $v'.color$  $\gets v.color$
        \State Add $v'$ to $\mathfrak{C}$
    \EndFor
    

        \State $R \gets$ all combinations with replacement of $|V(H)|$ elements from $n - |V(H)|$
        \For{each combination $r \in R$}
            \State $\mathfrak{C}' \gets$ copy of $\mathfrak{C}$
            \For{each $v \in r$}
                \State Create new vertex $v'$
                \State Set $v'.isTransversal$ $\gets False$
                \State Set $v'.color$  $\gets v.color$
                \State Add $v'$ to $\mathfrak{C}'$
            \EndFor
            \State Add $\mathfrak{C}'$ to $\mathcal{A}$
        \EndFor
    
    \State \Return $\mathcal{A}$
    \end{algorithmic}
\end{algorithm}

\section{Step 2: Constructing Kempe Chains}

For each coloring $\mathfrak{C}$, we now need to build graphs $G$ such that $H$ is isomorphic to their routing graph. 
To do so, for each edge of $H$, we try adding all possible paths joining the corresponding vertices of $T$
and alternating between their colors. To do so, we employ a recursive procedure that receives
the part of the graph that we have already finished, consisting of the paths for previously processed
edges of $H$ and of an initial segment $Q$ of the path we are constructing for the current edge $xy$ of $H$,
and lists all the ways of extending this path $Q$ to an alternating path ending in the vertex of $T$ corresponding
to $y$ and then adding the paths representing the remaining not yet processed edges of $H$

Let $s$ be the end of $Q$ and let $t$ be the vertex of $T$ corresponding to $y$. If $s \neq t$, then let $c$ be the color
corresponding to $y$ if $Q$ has even length and to $x$ if $Q$ has odd length, i.e., different from the color of $s$.
For each choice of a vertex $v\not\in V(Q)$ of color $c$, we extend the path $Q$ by adding the edge $sv$, then
recursively process the resulting graph. If $s=t$, we have just finished the path representing the edge $xy$;
we move on to the next edge $x'y'$ of $H$, replacing $Q$ by the trivial single-vertex path consisting of the vertex of $T$
corresponding to $x'$. Finally, if all edges of $H$ have already been processed, we report the current graph, whose
routing graph is necessarily isomorphic to $H$.

\begin{algorithm}[H]
    \caption{\textproc{BuildKempeChains}$(G, \mathit{chains}, s, t, \mathit{visited}, \mathit{available}, H)$}
    \begin{algorithmic}[1]
        \Statex \textbf{Input:}
        \begin{itemize}
            \item $G$ — adjacency map of the graph
            \item $\mathit{chains}$ — list of pairs $(s_i,t_i)$ to connect
            \item $s$ — end of current path
            \item $t$ — the vertex we want to reach by extending the current path
            \item $\mathit{visited}$ — all vertices $v$ such that the current path can potentially be extended by adding the edge $sv$.
            \item $\mathit{available}$ — list of available vertices to extend chain
            \item $H$ — minor to test existence against
        \end{itemize}
        \Statex \textbf{Procedure:}
        \If{$s = t$}
            \If{$\mathit{chains} = \emptyset $}
                \If{\textproc{TestMinor}($H$, $E(G)$, $V(H)$)}
                    \Return
                \Else
                    \State \textbf{report counterexample}
                \EndIf
            \Else \Comment{Building new chain}
                \State $(s',t') \gets \mathit{chains.extract()}$
                \State $c \gets$ color($t'$)
                \State $\mathit{available} \gets \{v \in V(G): \text{color}(v)=c\}$
                \State \Call{BuildKempeChains}{$G,\mathit{chains},s',t',\{s'\},\mathit{available},H$}
            \EndIf
            \State \Return
        \EndIf

        \ForAll{$v \in \mathit{available}$}
            \State add edge $(s,v)$ in $G$ 
            \State $\mathit{visited}' \gets \mathit{visited} \cup \{v\}$
            \State $\mathit{available}' \gets \{ w \in V(G): w \notin \mathit{visited}' \wedge \text{color}(w)=color(s)\}$
            \State \Call{BuildKempeChains}{$G,\mathit{chains},v,t,\mathit{visited}',\mathit{available}',H$}
            \State remove edge $(s,v)$ from $G$
        \EndFor
    \end{algorithmic}
\end{algorithm}

\section{Step 3: Checking for Rooted Minors}\label{sec:check-minor}
Finally, for each constructed graph~$G$, we check whether it contains~$H(G, \mathfrak{C}, T)$ as a $T$-rooted minor. If not, we have found a counter-example.

For this step, we use a heuristic tool for finding minor embeddings~\cite{dwavesystems2023minorminer}. The primary function we rely on is \texttt{find\_embedding()}, which implements the heuristic algorithm described in~\cite{cai2014practicalheuristicfindinggraph}.
Since \texttt{find\_embedding()} is a heuristic, we can be sure it is correct when it returns an embedding of~$H$ in~$G$. However, when it returns no embedding, we cannot immediately conclude that one does not exist.

To reduce the false negatives, if no embedding is found, we rerun the function 10 additional times with different random seeds. While the authors of the algorithm do not provide an estimate for the probability of false negatives (returning no embedding when one exists) or any 
probabilistic quantity that we can rely on, in our experience in most cases,
the function returns an embedding on the first try.

The algorithm is the following:

\begin{algorithm}[H]
    \caption{TestMinor($H$, $G$, $S$)}
    \begin{algorithmic}[1]
        \Statex \textbf{Input:} 
        \begin{itemize}
            \item \( H \) — the target minor graph
            \item \( G \) — the graph to look for the minor in
            \item \( S \) — a set of rooted vertices on which we want to get the rooted minor $H$ 
        \end{itemize}
        \Statex \textbf{Output:} True if $H$ is a rooted minor of $G$, otherwise False.

        \State $embedding \gets {find\_embedding}(H, G)$
        \If{$embedding \neq \emptyset$}
            \State \Return True
        \EndIf

        \For{$i \gets 1$ \textbf{to} 10}
            \State $seed \gets$ random integer
            \State $embedding \gets find\_embedding(H, G,$
            \Statex \hspace{\algorithmicindent}$\text{random\_seed} = seed, \text{suspend\_chains} = S)$
                        \If{$embedding \neq \emptyset$}
                \State \Return True
            \EndIf
        \EndFor

        \State \Return False
    \end{algorithmic}
\end{algorithm}

\section{Bringing everything together}
\label{main:algo:section}

We combine all the subroutines discussed in the previous paragraphs for the main algorithm. For a given graph  
$H$ and a number of vertices $n$ that each graph $G$ will have, we first generate all possible colorings for $n$ and $H$, where  
the transversal vertices $T$ of each coloring $\mathfrak{C}$ are mapped to the vertices of $H$. Then, for each such coloring $\mathfrak{C}$, we build graphs $G$ consisting only  
of Kempe chains, such that there is a Kempe chain between two transversal vertices of the coloring if and only if those two vertices  
form an edge in $H$. After constructing all Kempe chains, we test whether $H(G, \mathfrak{C}, T)$ is a $T$-rooted minor in $G$. If it is not,
then we know that $H$ is non-KM-forcing and that any supergraph of $H$ is also non-KM-forcing.

\begin{algorithm}[H]
    \caption{Searches for counter-examples}
    \label{alg:rooted_minor_search}
    \begin{algorithmic}[1]
      \Statex \textbf{Input:}
      \begin{itemize}
        \item $H$ — rooted minor we look for.
        \item $n$ — number of vertices in the graph $G$
      \end{itemize}
      \Statex \textbf{Output:} Graph $G$ where $H$ is isomorphic to the routing graph of $G$ with its corresponding coloring and transversal but not a rooted minor in $G$, or \textbf{None} if no such pair exists.
  
      \State $\textit{colorings} \gets \textsc{GenerateColorings}(H, n)$
      \ForAll{coloring $\mathfrak{C}$ \textbf{in} $\textit{colorings}$}
        \State $G \gets$ empty graph on $\mathfrak{C}$
        \State $result \gets \textsc{BuildKempeChains}(G, E(H), s, t, \emptyset, \mathit{available}, H)$
        \If{$result \neq \textbf{None}$}
          \State \Return $result$ \Comment{Found a counter-example}
        \EndIf
      \EndFor
  
      \State \Return \textbf{None} \Comment{No counter-examples found}
    \end{algorithmic}
  \end{algorithm}

\section{Results}
 
First, with the subroutine \textsc{TestMinor}, we verified the result of \cite{matthias_2022} that $K_7$ is non-KM-forcing.

We used the main algorithm from \ref{main:algo:section} to primarily test whether $K_6$ is non-KM-forcing or not.  
By \ref{thm:1}, finding any non-KM-forcing subgraph of $K_6$ is sufficient to prove that $K_6$ is non-KM-forcing.  
Since we also know that $K_4$ is KM-forcing \ref{thm:3} and that graphs with five vertices and at most six edges are KM-forcing,  
the subgraphs of $K_6$ we are looking for are between dense graphs on five vertices and graphs on six vertices.  
Moreover, we know that all cycles are KM-forcing \ref{thm:4}, so we do not need to consider $C_6$.  
Adding a pending edge to a KM-forcing graph still keeps this property \ref{thm:2}.  
So, we have a limited amount of subgraphs of $K_6$ to consider.

Because the algorithm has super-exponential time complexity, we could test for all candidate subgraphs $H$ of $K_6$ such that 
$H$ is isomorphic to $H(G, \mathfrak{C}, T)$, where $G$ is a graph with at most 13 vertices, $\mathfrak{C}$ is a coloring of $G$, and $T$ is a transversal such that $H(G,\mathfrak{C}, T)$ is isomorphic to $H$.

We found no counter-examples on such graphs.

\comment{I should add the other results here regarding other subgraphs of $K_7$, if there are any.}