\chapter{Introduction}

The four-color theorem, that every planar graph is four-colorable, was one of the central problems
in graph theory for over a century. Appel and Haken proved it in 1976 \cite{appel_haken_1977} using a computer program.
The proof was controversial because it was the first major theorem to be proved using a computer.
The theorem was later proved in 1997 by Robertson, Sanders, Seymour, and Thomas \cite{ROBERTSON19972},
still using a computer but with more straightforward configurations than Appel and Haken's in several aspects. 

Hadwiger's conjecture \cite{hadwiger_1943} suggests a generalization of the four-color theorem. It is considered one of 
the most challenging problems in graph theory. Before stating the conjecture, we define the notion of a \textit{graph minor}.

\begin{defn}[Minor]
    The $G$ contains a graph $H$ as a \textit{minor} if there exist pairwise disjoint subsets $\{S_1, S_2, \dots, S_{|V(H)|}\}$ of $V(G)$
    such that:
       \begin{itemize}
           \item For each $i$ from $1, 2, ... |V(H)|$, the subgraph of $G$ induced by vertices $S_i$, denoted $G[S_i]$, is connected
           \item For every edge $(u,v)$ in $H$, there is an edge of $G$ with one end in $S_u$ and the other end in $S_v$.
       \end{itemize} 
\end{defn}

\begin{conj} [Hadwiger 1943 \cite{hadwiger_1943}]
 For every integer $k \geq 0$, every graph $G$ with no $K_{k+1}$ minor can be colored with $k$ colors.
\end{conj}

Equivalently, it can be stated that every graph with chromatic number at least $k$ contains $K_k$ minor.

It was proved by Hadwiger himself for cases $k \leq 3$. Wagner \cite{wagner_1937} showed that the case $k = 4$ is equivalent to the four-color theorem.
\begin{enumerate}
    \item \textit{Forward direction:} The case $k = 4$ implies the four color theorem since by Wagner \cite{wagner_1937}, every planar graph has no $K_5$ or $K_{3, 3}$ as 
 minors, hence all planar graphs do not have $K_5$ as minor and Hadwiger's conjecture for case $k = 4$ claims if graph doesn't contain
    $K_5$ as minor, then it is $4$-colorable.
    \item \textit{Reverse direction:} Wagner \cite{wagner_1937} showed that a graph $G$ is $K_5$-minor free if it is obtained by 
clique sums of planar graphs and a Wagner graph $W$, where $W$ is a 3-colorable non-planar graph on eight vertices.
For any two four-colorable graphs, their clique sum is also four-colorable. Hence, the four-color theorem implies Hadwiger's conjecture
for the case $k = 4$.
\end{enumerate}

Robertson, Seymour, and Thomas proved the case $k = 5$ in 1993 \cite{robertson_seymour_1993}, 
where they did not use a computer to prove it. However, they proved that a minimal counter-example to the case $k = 5$ should have 
a vertex whose removal results in a planar graph, reducing the problem to the four-color theorem. The case $k = 6$ is still open, and there are some results in this direction as that of Albar and Gonçalves \cite{albar_goncalves_2013}. 
\begin{thm}
 Every graph with no $K_7$ minor is 8-colorable, and every graph with no $K_8$ minor is 10-colorable.
\end{thm}

Moreover, Kawarabayashi and Toft \cite{Kawarabayashi2005} proved that every 7-chromatic graph has $K_7$ or $K_{4,4}$ as minor.

Bollobás, Catlin, and Erdös \cite{BOLLOBAS1980195} showed that the conjecture is true for almost all graphs using
probabilistic arguments. In general, the cases $k \geq 6$ remain open.

Because of the conjecture's current state, there is interest in a weaker version known as the Linear Hadwiger's Conjecture.

\begin{conj} (Linear Hadwiger's Conjecture)
 There exists a constant $c$ such that, for every $k \geq 0$, every graph with no $K_{k+1}$ minor can be colored with $ck$ colors.
\end{conj}

For more than forty years, the best-known result for the linear version was that every graph with no $K_{t+1}$ minor is $O(t \sqrt{\log t})$-colorable.
This was proved in the 1980s by Kostochka \cite{Kostochka1984} and Thomason \cite{Thomason_1984} independently.
Recently, in 2025, Delcours and Postle \cite{JAMS2025} lowered this bound to $O(k \log \log(k))$.

\begin{defn}[Rooted minor]
    $H$ is a \textit{rooted minor} in $G$ if $H$ is a minor of $G$ and there exists a predeteremined set of vertices
    $\{x_i \in V(G) | i = 1, \dots, |V(H)|\}$, called the roots, such that $x_i$ is in $S_i$ for all $i$
\end{defn}

\begin{defn}[Colorful set]
 For a graph $G$, a set of vertices $S \subseteq V(G)$ is called \textit{colorful} in $G$ if for every chromatic coloring of $G$, $S$ contains at least one 
 vertex from each color of the coloring.
\end{defn}

The colorful sets of the graph are places that are 'hard to color'. Hence, we might hope to find the $K_{t}$ minors rooted in those sets. And this is 
exactly the Halroyd's conjecture.

\begin{conj} (Halroyd's conjecture)
 Let $G$ be a graph with chromatic number $k$, let $S$ be a colorful set in $G$, then $G$ has an S-rooted $K_k$ minor. 
\end{conj}

Holroyd called it the Strong Hadwiger Conjecture because it generalizes Hadwiger's conjecture. If we take $S = V(G)$,
then the Halroyd's conjecture states that graph $G$ has $S$-rooted $K_k$ minor, but since set $S$ is the all vertices of graph $G$, it means $G$ contains $K_k$ as a minor
which is exactly the statement of Hadwiger's conjecture.

Holroyd himself proved conjecture for cases $k \leq 3$ \cite{Holroyd1997}, and in 2024  the case
$k = 4$ was proved by Martinsson and Steiner \cite{MARTINSSON20241}.

A classical tool in studying Hadwiger's conjecture is the notion of Kempe chains. It was first introduced by Kempe in an attempt
to prove the four-color theorem. Even though his proof was not correct, Kempe chains were shown to be very useful in problems 
related to graph coloring.

\section{Kempe chains}

\begin{defn}[Kempe chain]
 Let $G$ be a graph with proper coloring. 
 For a vertex $v \in V(G)$ and two distinct colors $i$ and $j$ such that $v$ has either color $i$ or $j$, 
 the \textit{Kempe chain} containing $v$ with respect to colors $i$ and $j$ is the maximal connected subgraph
    $H$ of $G$ such that:
    \begin{enumerate}
        \item Every vertex $u \in V(H)$ has either color $i$ or $j$.
        \item For every edge $(u, w) \in E(H)$, both $u$ and $w$ are colored only with colors $i$ and $j$.
    \end{enumerate}
\end{defn}

We can observe that the vertices of colorful sets from Halroyd's conjecture must be connected by Kempe chains.
Suppose, for contradiction, that there exist two vertices  $u, v$ in a colorful set, colored  $i$ and  $j$ respectively,
but no Kempe chain of colors  $i$ and  $j$ connects them.
Then, consider the Kempe chain of colors  $i$ and  $j$ that contains  $u$. Swapping the colors within this chain—replacing every  $i$ with  $j$ and vice versa—produces a new proper coloring of the graph.
However, this would result in  $u$ taking the same color as  $v$, contradicting the assumption that the set is colorful.
Thus, every pair of vertices in a colorful set must be connected by a Kempe chain.

Since Hadwiger's conjecture is difficult to prove in general, it is interesting to study it for specific classes of graphs.
Hadwiger suggested looking into the graphs with a bounded number of optimal colorings \cite{hadwiger_1943}, 
one particular class is the uniquely optimally colorable graphs.

\begin{claim}
 Let $G$ be a uniquely $k$-colorable graph with colors $\{1, 2, \dots, k\}$ Let $v_1, v_2, \dots, v_k$ be differently colored vertices of the graph $G$, 
 where $v_i$ is has color $i$.
 Then there are Kempe chains between all pairs of $v_i$ and $v_j$ from $\{v_1, v_2, \dots, v_k\}$.
\end{claim}


\begin{proof}
 For any given pair of differently colored vertices $v_i$ and $v_j$, the induced subgraph of $G$ consisting of the vertices
 of $G$ with only colors of $i$ and $j$ is connected, if this wasn't the case
 then, we could color one of the disconnected components by swapping the colors $i$ and $j$, resulting in another proper coloring. 
 However, $G$ is uniquely k-colorable, hence a contradiction.
\end{proof}

This claim suggests a question: whether the existence of the Kempe chains forces an existence of $K_k$ minor rooted at
$\{v_1, v_2, \dots, v_k\}$.

Kriesell showed that there is a $K_k$ minor for 
cases $k \leq 10$ \cite{Kriesell2021}, if the graph is antitriangle-free \cite{Kriesell2017}, 
Moreover, with Mohr, they showed it is true for line graphs \cite{Kriesell2021}.

\section{Rooted minors}
One of the central problems in Graph Theory is to find a minor in a given graph. There has been significant progress 
in this direction, one of which is the structure theorem of Robertson and Seymour, which says that if a graph $G$ does not have
$K_t$ minor, then $G$ is ``almost embeddable'' on a surface of low Euler genus relative to $t$ \cite{RobertsonSeymourGM16}. This result leads Robertson and 
Seymour to prove Wagner's Conjecture. In the proof, they use the following theorems, which are proved in \cite{RobertsonSeymourGM20}

\begin{thm}
 let $G$ be a 3-connected graph and let $v_1, v_2, v_3$ be three distinct vertices. Then either $G$ has 
 five connected disjoint subgraphs $X_1, X_2, \dots, X_5$ such that $X_i$ contains $v_i$ for every $i = 1,2,3$
 and for every $j = 4,5$ $X_j$ has neighbour in each $X_i$ for all $i = 1,2,3$ or $G$ is planar such that $v_1, v_2, v_3$ are on boundary.
\end{thm}

\begin{thm}
 Let $G$ be a 4-connected graph and $v_1, v_2, v_3, v_4$ be four distinct vertices. Then either $G$ has 
 four connected disjoint subgraphs $X_1, X_2, \dots, X_4$ such that $X_i$ contains $v_i$ for every $i = 1,2,3,4$ and each
    $X_i$ has a neighbour in $X_j$ for every $i,j$ such that $i \neq j$ or $G$ is planar such that $v_1, v_2, v_3, v_4$ are on 
 the boundary.
\end{thm}

Those two results were the starting points for rooted minor problems. 
\newline
It turns out that rooted minors are not only
useful for the proof of Graph Minor theorem, but also for some structure theorems which are used to prove some existence
of graph minor, some of which are presented below:

\subsection*{Applications to Hadwiger's Conjecture}
Robertson, Seymour, and Thomas \cite{robertson_seymour_1993} used rooted $K_4$–minors to prove the case $k=5$ of Hadwiger's conjecture.
Kawarabayashi and Toft \cite{Kawarabayashi2005} used rooted minors to prove that every 7-chromatic graph has $K_7$ or $K_{4,4}$ as minor.

\subsection*{Embedded graphs and face covers}
In graphs embedded on surfaces, the concept of a rooted minor extends naturally to problems involving
face covers. A recent paper \cite{Fiorini2025} shows that in a 3-connected
graph embedded in a surface of Euler genus $g$, if the graph has no rooted $K_{2,t}$ minor,
then there exists a face cover whose size is bounded by a function of $g$ and $t$. 
In the planar case, they got $O(t^4)$ upper bound, which improved the result of Böhme and Mohar~\cite{BOHME2002291}.

\subsection*{Kempe chains and rooted minors}
Usually, in the context of Hadwiger's conjecture, only clique minors were considered. However, Kriesell and Mohr \cite{matthias_2022} considered
the following question, which does not necessarily look for clique minors. Let $G$ be a graph with a proper coloring
$\mathfrak{C}$, let $k = |\mathfrak{C}|$, and let $v_1, ..., v_{k}$ be a vertex set of $G$ with different colors.
Then, there is a system of Kempe chains for some pairs $(v_i, v_j)$.
They examined whether there is a rooted minor $H$ of $G$ on vertices $v_1, ..., v_{k}$, where $H$ has edges between 
$v_i, v_j$ if and only if there is a Kempe chain between $v_i$ and $v_j$ in $G$.

The answer to this question is affirmative for the case $k \leq 4$. For $k = 5$,
it holds for graphs with at most six edges but remains open in general.
The case $k = 6$ is also open, while counterexamples exist for $k \geq 7$.

In this paper, we investigate properties of minimal counterexample graphs for the case $k = 5$, where the answer is negative.
Additionally, we perform a computational enumeration for $k = 6$, examining all graphs with at most sixteen vertices.
Our results show that the answer remains positive for all graphs in this range.

