\chapter*{Introduction}
\addcontentsline{toc}{chapter}{Introduction}

The four-color theorem, that every planar graph is four-colorable, was one of the central problems
in graph theory for over a century. Appel and Haken proved it in 1976 \cite{appel_haken_1977} using a computer program.
The proof was controversial because it was the first major theorem to be proved using a computer.
The theorem was later proved in 1997 by Robertson, Sanders, Seymour, and Thomas \cite{ROBERTSON19972},
still using a computer, but with more straightforward configurations than Appel and Haken's in several aspects. 

Hadwiger's conjecture \cite{hadwiger_1943} suggests a generalization of the four-color theorem. It is considered one of 
the most challenging problems in graph theory. The conjecture is the following: 


\begin{conj} [Hadwiger 1943 \cite{hadwiger_1943} (Strong version)]
 For every integer $k \geq 0$, every graph $G$ with no $K_{k+1}$ minor can be colored with $k$ colors.
\end{conj}

It was proved by Hadwiger himself for cases $k \leq 3$. The case $k = 4$ is implied by the four-color theorem, since by Wagner's theorem \cite{wagner_1937} a graph is planar if and only if
it has no $K_5$ or $K_{3,3}$ minor. Robertson, Seymour, and Thomas proved the case $k = 5$ in 1993 \cite{robertson_seymour_1993}, 
where they did not use a computer to prove it. However, they proved that a minimal counter-example to the case $k = 5$ should have 
a vertex whose removal results in a planar graph, reducing the problem to the four-color theorem. $k = 6$ is still open, and there are some result in this direction is that of Albar and Gonçalves \cite{albar_goncalves_2013}. 
\begin{thm}
 Every graph with no $K_7$ minor is 8-colorable, and every graph with no $K_8$ minor is 10-colorable.
\end{thm}

And Kawarabayashi and Toft \cite{Kawarabayashi2005} proved that every 7-coloarable graph has $K_7$ or $K_{4,4}$ as minor.

Bollobás, Catlin, and Erdös \cite{BOLLOBAS1980195} showed that the conjecture is true for almost all graphs using
probabilistic arguments. In general, the cases $k \geq 6$ remain open.

Because of the current state of the conjecture, there is an interest in a weaker version of the conjecture.

\begin{conj} (Hadwiger's Conjecture weak version)
    There exists a constant $c$ such that, for every $k \geq 0$, every graph with no $K_{k+1}$ minor can be colored with $ck$ colors.
\end{conj}

The best-known result for the weak version is that for every graph with no $K_{t+1}$ minor, there exists a constant $c$ such that the graph is $ck\sqrt{\log{k}}$-colorable.
This was proved forty years ago by Kostochka \cite{Kostochka1984} and Thomason \cite{Thomason_1984} independently. Moreover, there has been no significant improvement since then.

Since the conjecture is hard to prove in general, it is interesting to study the conjecture for specific classes of graphs.
Hadwiger suggested looking into the graphs with a bounded number of optimal colorings \cite{hadwiger_1943}, 
one particular class is the uniquely optimally colorable graphs.

\begin{claim}
    Let $x_1, x_2, ..., x_k$ be differently colored vertices of a uniquely $k-colorable$ graph $G$. Then there is
    system of edge-disjoint $x_{i}, x_{j}$ paths $(\forall i, j \in [n], i \neq j)$, 
    by taking only the edges between the corresponding two color classes. 
\end{claim}


\begin{proof}
    For any given $i, j \in [n], i \neq j$, the graph $G[V_i \cup V_j]$ is a connected 
    component, where $V_i$ and $V_j$ are the color classes of $x_i$ and $x_j$ respectively, if this wasn't the case
    then, we could color one of the disconnected components by swapping the colors $i$ and $j$, resulting to another coloring. 
    However, $G$ is uniquely k-colorable, hence a contradiction.
\end{proof}

The question is whether a $K_k$ minor exists rooted at $x_1, ..., x_k$. Kriesell showed that the answer is positive for 
cases $k \leq 10$ \cite{Kriesell2021}, if the graph is antitriangle-free \cite{Kriesell2017}, 
Moreover, with Mohr, they showed it is true for line graphs \cite{Kriesell2021}.


Taking a vertex set of the graph containing all colors seems a natural approach to look for a $K_{t}$ minor, since it contains
all the colors of the graph, it's a place which is 'hard to color', hence it's a good place to hope for finding a minor. And this is 
exactly the Halroyd's conjecture.

First let's define what is a colorful set.

\begin{defn}
    For graph $G$, A set of vertices $S \subseteq V(G)$ is called colorful if for every proper coloring of $G$, $S$ contains atleast one 
    vertex for each color of the coloring.
\end{defn}

\begin{conj} (Halroyd's conjecture)
    For every $t \geq 0$, every graph $G$ with no rooted $K_{t+1}$ minor has a colorful set of size $t$.
\end{conj}

Holroyd called it the Strong Hadwiger Conjecture since if we take $S = V(G)$, it implies Hadwiger's conjecture.

Holroyd himself has proven the conjecture for cases $k \leq 3$ \cite{Holroyd1997}, and recently it has been proved for the case
$k = 4$ by Martinsson and Steiner \cite{MARTINSSON20241}.


Usually, in the context of Hadwiger's conjecture, only clique minors were considered; however, Kriesell and Mohr \cite{matthias_2022} considered
the following question, which does not necessarily look for clique minors. Let $G$ be a graph with a proper coloring
$\mathcal{C}$, let $k = |\mathcal{C}|$, and let $x_1, ..., x_{k}$ be a vertex set of $G$ with different colors.
Then there is a system of esdge-disjoint $(x_i, x_j)$-paths for some pairs $(x_i, x_j)$.
They examined weather there is a rooted minor $H$ of $G$ on vertices $x_1, ..., x_{k}$, where $H$ has edges between 
$x_i, x_j$ if and only if there is an edge-disjoint path between $x_i$ and $x_j$ in $G$.

The answer to this question is affirmative for the cases of $k \leq 4$; for $k = 5$, it is true for graphs
with at most six edges, but in general open, and for $k = 6$, it's open, and it is negative for $k \geq 7$.

This paper will characterize minimal counterexample graphs for $k = 5$ such that the answer is negative. 
We also made a computer enumeration for the case $k = 6$, checking all the graphs with at most sixteen vertices, and 
we found that the answer was positive for all of them.

