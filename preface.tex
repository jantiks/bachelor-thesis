\chapter{Introduction}

The four-color theorem, that every planar graph is four-colorable, was one of the central problems
in graph theory for over a century. Appel and Haken proved it in 1976 \cite{appel_haken_1977} using a computer program.
The proof was controversial because it was the first major theorem to be proved using a computer.
The theorem was later proved in 1997 by Robertson, Sanders, Seymour, and Thomas \cite{ROBERTSON19972},
still using a computer but with more straightforward configurations than Appel and Haken's in several aspects. 

Hadwiger's conjecture \cite{hadwiger_1943} suggests a generalization of the four-color theorem. It is considered one of 
the most challenging problems in graph theory. Before stating the conjecture, we define the notion of a \textit{graph minor}.

\begin{defn}[Minor]
    The $G$ contains a graph $H$ as a \textit{minor} if there exist a system $\{B_v : v \in V(H)\}$ of pairwise disjoint subsets of $V(G)$
    such that:
       \begin{itemize}
           \item For every vertex $v$ from $V(H)$, the subgraph of $G$ induced by vertices $B_v$, denoted $G[B_v]$, is connected.
           \item For every edge $u,v$ in $E(H)$, there is an edge of $G$ with one end in $G[B_u]$ and the other end in $G[B_v]$.
       \end{itemize} 

       We say that the system $\{B_v : v \in V(H)\}$ is a model of $H$ in $G$.
\end{defn}

\begin{conj} [Hadwiger 1943 \cite{hadwiger_1943}]
 For every integer $k \geq 0$, every graph $G$ with no $K_{k+1}$ minor can be colored with $k$ colors.
\end{conj}

Equivalently, it can be stated that every graph with chromatic number at least $k$ contains $K_k$ minor.

It was proved by Hadwiger himself for cases $k \leq 3$. Wagner \cite{wagner_1937} showed that the case $k = 4$ is equivalent to the four-color theorem.
\begin{enumerate}
    \item \textit{Forward direction:} The case $k = 4$ implies the four color theorem since by Wagner \cite{wagner_1937}, every planar graph has no $K_5$ or $K_{3, 3}$ as 
 minors. Hence all planar graphs do not have $K_5$ as minor. Hadwiger's conjecture for case $k = 4$ claims if graph doesn't contain
    $K_5$ as minor, then it is $4$-colorable.
    \item \textit{Reverse direction:} Wagner \cite{wagner_1937} showed that a graph $G$ is $K_5$-minor free if it is obtained by 
clique sums of planar graphs and a Wagner graph $W$, where $W$ is a 3-colorable non-planar graph on eight vertices.
For any two four-colorable graphs, their clique sum is also four-colorable. Hence, the four-color theorem implies Hadwiger's conjecture
for the case $k = 4$.
\end{enumerate}

Robertson, Seymour, and Thomas proved the case $k = 5$ in 1993 \cite{robertson_seymour_1993}, 
where they did not use a computer to prove it. However, they proved that a minimal counter-example to the case $k = 5$ should have 
a vertex whose removal results in a planar graph, reducing the problem to the four-color theorem. The case $k = 6$ is still open, and there are some results in this direction as that of Albar and Gonçalves \cite{albar_goncalves_2013}. 
\begin{thm}
 Every graph with no $K_7$ minor is 8-colorable, and every graph with no $K_8$ minor is 10-colorable.
\end{thm}

Moreover, Kawarabayashi and Toft \cite{Kawarabayashi2005} proved that every 7-chromatic graph has $K_7$ or $K_{4,4}$ as minor.
Bollobás, Catlin, and Erdös \cite{BOLLOBAS1980195} showed that the conjecture is true for almost all graphs using
probabilistic arguments. In general, the cases $k \geq 6$ remain open.
Because of the conjecture's current state, there is interest in a weaker version known as the Linear Hadwiger's Conjecture.

\begin{conj} (Linear Hadwiger's Conjecture)
 There exists a constant $c$ such that, for every $k \geq 0$, every graph with no $K_{k+1}$ minor can be colored with $ck$ colors.
\end{conj}

For more than forty years, the best-known result for the linear version was that every graph with no $K_{t+1}$ minor is $O(t \sqrt{\log t})$-colorable.
This was proved in the 1980s by Kostochka \cite{Kostochka1984} and Thomason \cite{Thomason_1984} independently.
Recently, in 2025, Delcours and Postle \cite{JAMS2025} lowered this bound to $O(k \log \log(k))$.

Holroyd\cite{Holroyd1997} tryied to strengthen Hadwiger’s conjecture by looking specific regions in a graph where a minor of a complete graph is likely to appear, based on the coloring of the region.
Before stating his conjecture, we need to define the concepts of rooted minors and colorful sets.

\begin{defn}[Rooted minor]
    Let $G$ and $H$ be graphs and let $S = \{X_v : v \in V(H)\}$ be a system of distinct vertices of $G$. Then $H$
    is an $S$-rooted minor of $G$ if exists a model $\{B_v : v \in V(H)\}$ of $H$ in $G$ such that for every
    vertex $v \in V(H)$ $X_v \in B_v$.
\end{defn}

\begin{defn}[Chromatic coloring]
    Let $G$ be a graph. The chromatic number of $G$ denoted as $\chi(G)$ is the smallest integer $k$ such that $G$ is 
    properly colored with $k$ colors. We say that a coloring $\mathfrak{C}$ is chromatic coloring of the graph $G$ if 
    it has exactly $\chi(G)$ color classes.
\end{defn}

\begin{defn}[Colorful set]
 For a graph $G$, a set of vertices $S \subseteq V(G)$ is called \textit{colorful} in $G$ if for every chromatic coloring of $G$, $S$ contains at least one 
 vertex from each color of the coloring.
\end{defn}

The colorful sets of the graph are places that are 'hard to color'. Hence, we might hope to find the $K_{t}$ minors rooted in those sets. And this is 
exactly the Halroyd's conjecture.

\begin{conj} (Halroyd's conjecture)
 Let $G$ be a graph with chromatic number $k$, let $S$ be a colorful set in $G$. Then there exists
 a subset $S' \subseteq S$ of size $k$ such that $G$ has an $S'$-rooted $K_k$ minor. 
\end{conj}

\comment{In notes I have: Formulate that the system doesn't matter for complete graphs for colorful sets. 
I am not sure what we wanted to say there}

Holroyd called it the Strong Hadwiger Conjecture because it generalizes Hadwiger's conjecture. If we take $S = V(G)$,
then the Halroyd's conjecture states that there exists an $S' \subseteq S$ of size $k$ such that the graph $G$ has $S'$-rooted $K_k$ minor. 
Since the set $S$ is the all vertices of graph $G$, it means $G$ contains $K_k$ as a minor
which is exactly the statement of Hadwiger's conjecture.

Holroyd himself proved conjecture for cases $k \leq 3$ \cite{Holroyd1997}, and in 2024  the case
$k = 4$ was proved by Martinsson and Steiner \cite{MARTINSSON20241}.

A classical tool in studying Hadwiger's conjecture is the notion of Kempe chains. It was first introduced by Kempe in an attempt
to prove the four-color theorem. Even though his proof was not correct, Kempe chains were shown to be very useful in problems 
related to Hadwiger’s conjecture. They were used for proving the four-color and five-color theorems as well.

\section{Kempe chains}

\begin{defn}[Proper coloring]
    A proper (vertex) coloring of a graph $G$ is a coloring of the vertices of $G$ such that no two adjacent vertices
    share the same color.
\end{defn}

\comment{Should we define proper coloring earlier, and also say that we assume that all colorings are proper? Because
by default we assume it throughout the paper}

\begin{defn}[Kempe chain]
 Let $G$ be a graph with proper coloring. Then for two distinct color classes $i$ and $j$, the Kempe chain 
 in colors $i$ and $j$ is the maximal connected subgraph of $G$ where vertices have only colors of $i$ or $j$.
\end{defn}

\begin{claim}
    \label{colorful_kempe}
    Let $G$ be a graph with proper coloring and let $S$ be a colorful set in $G$. For a color class $i$, let 
    $S_i$ be the set of vertices from $S$ that have color $i$. Then for every distinct color classes $i$ and $j$
    There exists a Kempe chain between some of the vertices of $S_i$ and $S_j$.
\end{claim}
\begin{proof}
    Assume for contradiction that there is no Kempe chain between $S_i$ and $S_j$.  We will show that we can recolor the vertices of $S_i$ with color $j$ and still maintain a proper coloring, contradicting the assumption that $S$ is colorful (since in the new coloring, no vertex of $S$ has color $i$).
    
    \begin{enumerate}
      \item Let $v\in S_i$.
        \begin{itemize}
          \item If $v$ is not contained in any Kempe chain of colors $i$ and $j$.
          \item Otherwise, $v$ lies in some Kempe chain of colors $i$ and $j$.  By our assumption, that Kempe chain contains no vertices of $S_j$ we can switch the colors $i\leftrightarrow j$ along that chain. The resulting coloring is proper, where $v$ now has color $j$.
        \end{itemize}
      \item Repeat this for each vertex of $S_i$.  At each step, properness is maintained, and no vertex of $S_j$ is ever touched (since there are no $i$–$j$ chains reaching $S_j$).  After recoloring all of $S_i$, the colorful set $S$ no longer contains any vertex of color $i$, a contradiction.
    \end{enumerate}
    
    Hence there exists a Kempe chain connecting some vertex in $S_i$ to some vertex in $S_j$.
    \end{proof}
    
We can observe that the vertices of colorful sets from Halroyd's conjecture must be connected by Kempe chains.
Suppose, for contradiction, that there exist two vertices  $u, v$ in a colorful set, colored  $i$ and  $j$ respectively,
but no Kempe chain of colors  $i$ and  $j$ connects them.
Then, consider the Kempe chain of colors  $i$ and  $j$ that contains  $u$. Swapping the colors within this chain—replacing every  $i$ with  $j$ and vice versa—produces a new proper coloring of the graph.
However, this would result in  $u$ taking the same color as  $v$, contradicting the assumption that the set is colorful.
Thus, every pair of vertices in a colorful set must be connected by a Kempe chain.

Since Hadwiger's conjecture is difficult to prove in general, it is interesting to study it for specific classes of graphs.
Hadwiger suggested looking into the graphs with a bounded number of optimal colorings \cite{hadwiger_1943}, 
one particular class is the uniquely optimally colorable graphs.

\begin{claim}
 Let $G$ be a uniquely $k$-colorable graph with colors $\{1, 2, \dots, k\}$ Let $v_1, v_2, \dots, v_k$ be differently colored vertices of the graph $G$, 
 where $v_i$ is has color $i$.
 Then there are Kempe chains between all pairs of $v_i$ and $v_j$ from $\{v_1, v_2, \dots, v_k\}$.
\end{claim}


\begin{proof}
    Let $S := \{v_1, v_2, \dots, v_k\}$, then $S$ is a colorful set. By the claim \ref{colorful_kempe} for every distinct color classes
    $i$ and $j$ there exists a Kempe chain between $S_i$ and $S_j$. Since each color appears only once in $S$, we have that
    each $S_i$ has size of 1. Hence, there is a Kempe chain between each distinct pair of the vertices of the colorful set.
\end{proof}

This claim suggests a question: whether the existence of the Kempe chains forces an existence of $K_k$ minor rooted at
$\{v_1, v_2, \dots, v_k\}$.

Kriesell proved the Hadwiger’s conjecture for uniquely $k$-colorable graphs where $k \leq 10$ \cite{Kriesell2021}, if the graph is antitriangle-free \cite{Kriesell2017}.
Moreover, with Mohr, they proved it is true for line graphs \cite{Kriesell2019}.

\section{Rooted minors}
One of the central problems in Graph Theory is to find a minor in a given graph. There has been significant progress 
in this direction, one of which is the structure theorem of Robertson and Seymour, which says that if a graph $G$ does not have
$K_t$ minor, then $G$ is ``almost embeddable'' on a surface of low Euler genus relative to $t$ \cite{RobertsonSeymourGM16}.
This result was developed as part of their proof of Wagner's Conjecture—now known as the Robertson–Seymour Theorem
—which says that the class of finite undirected graphs is well-quasi-ordered under the graph minor relation.
In the proof, they use the following theorems, which are proved in \cite{RobertsonSeymourGM20}

\begin{thm}
 let $G$ be a 3-connected graph and let $v_1, v_2, v_3$ be three distinct vertices. Then either $G$ has 
 five connected disjoint subgraphs $X_1, X_2, \dots, X_5$ such that $X_i$ contains $v_i$ for every $i = 1,2,3$
 and for every $j = 4,5$ $X_j$ has neighbour in each $X_i$ for all $i = 1,2,3$ or $G$ is planar such that $v_1, v_2, v_3$ are on boundary.
\end{thm}

\begin{thm}
 Let $G$ be a 4-connected graph and $v_1, v_2, v_3, v_4$ be four distinct vertices. Then either $G$ has $K_4$ minor rooted
 at $\{v_1, v_2, v_3, v_4\}$ or $G$ is planar such that $v_1, v_2, v_3, v_4$ are on the boundary.
\end{thm}

Those two results were the starting points for rooted minor problems. 
\newline
It turns out that rooted minors are not only
useful for the proof of Graph Minor theorem, but also for some structure theorems which are used to prove some existence
of graph minor, some of which are presented below:

Robertson, Seymour, and Thomas \cite{robertson_seymour_1993} used rooted $K_4$–minors to prove the case $k=5$ of Hadwiger's conjecture.
Kawarabayashi and Toft \cite{Kawarabayashi2005} used rooted minors to prove that every 7-chromatic graph has $K_7$ or $K_{4,4}$ as minor.

In graphs embedded on surfaces, the concept of a rooted minor extends naturally to problems involving
face covers. A recent paper \cite{Fiorini2025} shows that in a 3-connected
graph embedded in a surface of Euler genus $g$, if the graph has no rooted $K_{2,t}$ minor,
then there exists a face cover whose size is bounded by a function of $g$ and $t$. 
In the planar case, they got $O(t^4)$ upper bound, which improved the result of Böhme and Mohar~\cite{BOHME2002291}.

\subsection*{Kempe chains and rooted minors}
Usually, in the context of Hadwiger's conjecture, only clique minors were considered. However, Kriesell and Mohr \cite{matthias_2022} considered
the following question, which does not necessarily look for clique minors. Let $G$ be a graph with a proper coloring
$\mathfrak{C}$, let $k = |\mathfrak{C}|$, and let $v_1, ..., v_{k}$ be a vertex set of $G$ with different colors.
Then, there is a system of Kempe chains for some pairs $(v_i, v_j)$.
They examined whether there is a rooted minor $H$ of $G$ on vertices $v_1, ..., v_{k}$, where $H$ has edges between 
$v_i, v_j$ if and only if there is a Kempe chain between $v_i$ and $v_j$ in $G$.

The answer to this question is affirmative for the case $k \leq 4$. For $k = 5$,
it holds for graphs with at most six edges but remains open in general.
The case $k = 6$ is also open, while counterexamples exist for $k \geq 7$.

In this paper, we investigate properties of minimal counterexample graphs for the case $k = 5$.
Additionally, we perform a computational enumeration for $k = 6$, examining all graphs with at most sixteen vertices.
Our results show that the answer remains positive for all graphs in this range.

