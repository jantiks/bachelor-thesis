\chapter{Introduction}

The four-color theorem, that every planar graph is four-colorable, was one of the central problems
in graph theory for over a century. Appel and Haken proved it in 1976 \cite{appel_haken_1977} using a computer program.
The proof was controversial because it was the first major theorem to be proved using a computer.
The theorem was later proved in 1997 by Robertson, Sanders, Seymour, and Thomas \cite{ROBERTSON19972},
still using a computer but with more straightforward configurations than Appel and Haken's in several aspects. 

Hadwiger's conjecture \cite{hadwiger_1943} suggests a generalization of the four-color theorem. It is considered one of 
the most challenging problems in graph theory. Before stating the conjecture, we define the notion of a \textit{graph minor}.

\begin{defn}[Minor]
 The graph $G$ contains a graph $H$ as a \emph{minor} if there exist a system $\{B_v : v \in V(H)\}$ of pairwise disjoint subsets of $V(G)$
 such that:
       \begin{itemize}
           \item For every vertex $v$ from $V(H)$, the subgraph of $G$ induced by vertices $B_v$, denoted $G[B_v]$, is connected.
           \item For every edge $u,v$ in $E(H)$, there is an edge of $G$ with one end in $G[B_u]$ and the other end in $G[B_v]$.
       \end{itemize} 

 We say that the system $\{B_v : v \in V(H)\}$ is a model of $H$ in $G$.
\end{defn}

\begin{conj} [Hadwiger 1943 \cite{hadwiger_1943}]
 For every integer $k \geq 0$, every graph $G$ with no $K_{k+1}$ minor can be colored with $k$ colors.
\end{conj}

Equivalently, every graph with a chromatic number of at least $k$ contains $K_k$ minor.

\begin{rem}
   Throughout this paper, all colorings are assumed to be proper, i.e., adjacent vertices receive different colors.
\end{rem}

The conjecture was proved by Hadwiger himself for cases $k \leq 3$. Wagner \cite{wagner_1937} showed that the case $k = 4$ is equivalent to the four-color theorem.
\begin{enumerate}
    \item \textit{Forward direction:} The case $k = 4$ implies the four color theorem since by Wagner \cite{wagner_1937}, every planar graph has no $K_5$ or $K_{3, 3}$ as 
 minors. Hence, all planar graphs do not have $K_5$ as minor. Hadwiger's conjecture for case $k = 4$ claims if the graph does not contain
    $K_5$ as minor, then it is $4$-colorable.
    \item \textit{Reverse direction:} Wagner \cite{wagner_1937} showed that a graph $G$ is $K_5$-minor free if it is obtained by 
clique sums of planar graphs and a Wagner graph $W$, where $W$ is a 3-colorable non-planar graph on eight vertices.
For any two four-colorable graphs, their clique sum is also four-colorable. Hence, the four-color theorem implies Hadwiger's conjecture
for the case $k = 4$.
\end{enumerate}

Robertson, Seymour, and Thomas proved the case $k = 5$ in 1993 \cite{robertson_seymour_1993}, 
where they did not use a computer to prove it. However, they proved that a minimal counter-example to the case $k = 5$ should have 
a vertex whose removal results in a planar graph, reducing the problem to the four-color theorem. The case $k = 6$ is still open, and there are some results in this direction as that of Albar and Gonçalves \cite{albar_goncalves_2013}. 
\begin{thm}
\label{thm:albargoncalves}
 Every graph with no $K_7$ minor is 8-colorable, and every graph with no $K_8$ minor is 10-colorable.
\end{thm}

Moreover, Kawarabayashi and Toft \cite{Kawarabayashi2005} proved that every 7-chromatic graph has $K_7$ or $K_{4,4}$ as minor.
Bollobás, Catlin, and Erdös \cite{BOLLOBAS1980195} showed that the conjecture is true for almost all graphs using
probabilistic arguments. In general, the cases $k \geq 6$ remain open.
Because of the conjecture's current state, there is interest in a weaker version known as the Linear Hadwiger's Conjecture.

\begin{conj} (Linear Hadwiger's Conjecture)
 There exists a constant $c$ such that, for every $k \geq 0$, every graph with no $K_{k+1}$ minor can be colored with $ck$ colors.
\end{conj}

For more than forty years, the best-known result for the linear version was that every graph with no $K_{t+1}$ minor is $O(t \sqrt{\log t})$-colorable.
This was proved in the 1980s by Kostochka \cite{Kostochka1984} and Thomason \cite{Thomason_1984} independently.
Recently, in 2025, Delcours and Postle \cite{JAMS2025} lowered this bound to $O(k \log \log(k))$.

Holroyd\cite{Holroyd1997} tried to strengthen Hadwiger's conjecture by looking at specific regions in a graph where a minor of a complete graph is likely to appear based on the coloring of the region.
Before stating his conjecture, we must define the concepts of rooted minors and colorful sets.

\begin{defn}[Rooted minor]
 Let $G$ and $H$ be graphs and let $S = \{x_v: v \in V(H)\}$ be a system of distinct vertices of $G$. Then $H$
 is an \emph{$S$-rooted minor} of $G$ if exists a model $\{B_v : v \in V(H)\}$ of $H$ in $G$ such that for every
 vertex $v \in V(H)$ $x_v \in B_v$.
\end{defn}

\begin{defn}[Chromatic coloring]
 Let $G$ be a graph. The chromatic number of $G$ denoted as $\chi(G)$ is the smallest integer $k$ such that $G$ is 
 properly colored with $k$ colors. We say that a coloring $\mathfrak{C}$ is \emph{chromatic coloring} of the graph $G$ if 
 it has exactly $\chi(G)$ color classes.
\end{defn}

\begin{defn}[Colorful set]
 For a graph $G$, a set of vertices $S \subseteq V(G)$ is called \emph{colorful} in $G$ if for every chromatic coloring of $G$, $S$ contains at least one 
 vertex from each color of the coloring.
\end{defn}

The colorful sets of the graph are places that are 'hard to color.' Hence, we might find the $K_{t}$ minors rooted in those sets. And this is 
exactly the Holroyd's conjecture.

\begin{conj} (Holroyd's conjecture)
 Let $G$ be a graph with chromatic number $k$, and let $S$ be a colorful set in $G$. Then there exists
 a subset $ S' \subseteq S$ of size $k$ such that $G$ contains an $S'$-rooted $K_k$ minor. 
\end{conj}


\begin{rem}
   In the case of rooted clique‐minors $K_k$, the labels of the vertices of $K_k$ play no role,
   since any two bijections $S'\to V(K_k)$ differ by a permutation of the bags and hence give isomorphic models.
   Equivalently, once we have chosen the colourful set $S'$ of size $k$,
   there is no need to specify which $x_v\in S'$ goes into which bag: 
   any assignment results in an $S'$‐rooted $K_k$‐minor.
 \end{rem}


Holroyd called it the Strong Hadwiger Conjecture because it generalizes Hadwiger's conjecture. If we take $S = V(G)$,
then the Holroyd's conjecture states that there exists an $S' \subseteq S$ of size $k$ such that the graph $G$ has $S'$-rooted $K_k$ minor. 
Since the set $S$ is the all vertices of graph $G$, it means $G$ contains $K_k$ as a minor
which is exactly the statement of Hadwiger's conjecture.

Holroyd himself proved the conjecture for cases $k \leq 3$ \cite{Holroyd1997}, and in 2024  the case
$k = 4$ was proved by Martinsson and Steiner \cite{MARTINSSON20241}.

The notion of Kempe chains is a classical tool in studying Hadwiger's conjecture. Kempe first introduced it in an attempt
to prove the four-color theorem. Even though his proof was incorrect, they were later used to prove the four-color and five-color theorems.
Kempe chains show to be very useful in problems related to Hadwiger's conjecture.

\section{Kempe chains}

\begin{defn}[Kempe chain]
 Let $G$ be a graph with a coloring. Then, for two distinct color classes, $i$, and $j$, the \emph{Kempe chain}
 in colors $i$ and $j$ is the maximal connected subgraph of $G$ where vertices have only colors of $i$ or $j$.
\end{defn}

\begin{lemma}
    \label{colorful_kempe}
 Let $G$ be a graph with a coloring and let $S$ be a colorful set in $G$. For a color class $i$, let 
    $S_i$ be the set of vertices from $S$ that have color $i$. Then, for every distinct color class $i$ and $j$
 a Kempe chain exists between some vertices of $S_i$ and $S_j$.
\end{lemma}
\begin{proof}
 Assume for contradiction that there is no Kempe chain between $S_i$ and $S_j$. We will show that we can recolor the vertices of $S_i$ with color $j$ and still maintain a proper coloring, contradicting the assumption that $S$ is colorful (since in the new coloring, no vertex of $S$ has color $i$).
    
    \begin{enumerate}
      \item Let $v\in S_i$.
        \begin{itemize}
          \item If $v$ is not contained in any Kempe chain of colors $i$ and $j$, switch the color of $v$ to $j$.
          \item Otherwise, $v$ lies in some Kempe chain of colors $i$ and $j$.  Assuming the Kempe chain contains no vertices of $S_j$, we can switch the colors $i\leftrightarrow j$ along that chain. The resulting coloring is proper, where $v$ now has color $j$.
        \end{itemize}
      \item Repeat this for each vertex of $S_i$. The coloring is maintained at each step, and no vertex of $S_j$ is ever recolored (since there are no $i$–$j$ chains reaching~$S_j$). After recoloring all of $S_i$, the colorful set $S$ no longer contains any vertex of color $i$, a contradiction.
    \end{enumerate}
    
 Hence, a Kempe chain exists connecting some vertex in $S_i$ to some vertex in~$S_j$.
    \end{proof}

    Since Hadwiger's conjecture is difficult to prove in general, it is interesting to study it for specific classes of graphs.

    \begin{defn}
    \label{def:uniquely_colorable}
    A graph $G$ is called \emph{uniquely optimally colorable} if it has only one chromatic coloring up to permutation of the colors.
    \end{defn}
    
    Hadwiger suggested looking into the graphs with a bounded number of optimal colorings \cite{hadwiger_1943}.
    One particular class is the uniquely optimally colorable graphs,
    which Kriesell proved for uniquely $k$-colorable graphs where $k \leq 10$ \cite{Kriesell2021}.
    Kriesell also proved Hadwiger's conjecture for the antitriangle-free graphs \cite{Kriesell2017}.
    Moreover, with Mohr, they proved it is true for line graphs \cite{Kriesell2019}.

    \begin{claim}
 Let $G$ be a uniquely $k$-colorable graph with colors $\{1, 2, \dots, k\}$ Let $v_1, v_2, \dots, v_k$ be differently colored vertices of the graph $G$, 
 where $v_i$ is has color $i$.
 Then there are Kempe chains between all pairs of $v_i$ and $v_j$ from $\{v_1, v_2, \dots, v_k\}$.
    \end{claim}
    
    
    \begin{proof}
 Let $S := \{v_1, v_2, \dots, v_k\}$, then $S$ is a colorful set. By the lemma \ref{colorful_kempe} for every distinct color classes
        $i$ and $j$ there exists a Kempe chain between $S_i$ and $S_j$. Since each color appears only once in $S$, 
 each $S_i$ has a size of 1. Hence, there is a Kempe chain between each distinct pair of the vertices of the colorful set.
    \end{proof}

This shows a structural connection between the colorful sets and the Kempe chains. 
It is a natural question to ask whether the existence of Kempe chains
between all pairs of vertices of a colorful set $S$ implies the existence of a $K_k$ minor rooted at $S$.
However, Kriesell and Mohr \cite{matthias_2022} showed that this is not true in general.

Usually, in the context of Hadwiger's conjecture, only clique minors were considered. However, Kriesell and Mohr \cite{matthias_2022} considered
the following question, which does not necessarily look for clique minors. Let $G$ be a graph with a coloring
$\mathfrak{C}$, let $k = |\mathfrak{C}|$, and let $v_1, ..., v_{k}$ be a vertex set of $G$ with different colors.
Then, there is a system of Kempe chains for some pairs $(v_i, v_j)$.
Let $H$ be a graph with vertices $v_1, ..., v_{k}$ and edges between $v_i, v_j$ if and only if there is a Kempe chain between $v_i$ and $v_j$ in $G$.
They questioned whether there is a rooted minor $H$ of $G$ on vertices $v_1, ..., v_{k}$.

The answer to this question is affirmative for the case $k \leq 4$. For $k = 5$,
it holds for graphs with at most six edges but generally remains open.
The case $k = 6$ is also open, while counter-examples exist for $k \geq 7$.

In the next chapter, \ref{chap:background}, we present the main results of Kriesell and Mohr~\cite{matthias_2022}, 
where they investigate the above question. We discuss the main results from it and set up the necessary 
terminology for us to do our own investigations. In Chapter~\ref{chap:structural-properties}, 
we investigate connectivity properties of minimal counter-example graphs for the case $k = 5$. 
In chapter~\ref{chap:enumeration}, we develop an algorithm which performs computer enumeration for finding 
counter-examples within graphs with at most 13 vertices, Our results show that there are no counter-examples for $k = 6$ within this range.
We use the algorithm to verify the result of Kriesell and Mohr \cite{matthias_2022} for the case $k = 7$.


Before moving to the next chapter, we will present some results regarding the rooted minors and their applications in graph theory. 

\section{Rooted minors}
\label{section:rooted-minors}
One of the central problems in graph theory is finding a minor in a given graph. There has been significant progress 
in this direction, one of which is the structure theorem of Robertson and Seymour, which says that if a graph $G$ does not have
$K_t$ minor, then $G$ is ``almost embeddable'' on a surface of low Euler genus relative to $t$ \cite{RobertsonSeymourGM16}.
This result was developed as part of their proof of Wagner's Conjecture—now known as the Robertson–Seymour Theorem
—which says that the class of finite undirected graphs is well-quasi-ordered under the graph minor relation.
In the proof, they use the following theorems, which are proved in \cite{RobertsonSeymourGM20}

\begin{thm}
 let $G$ be a 3-connected graph and let $v_1, v_2, v_3$ be three distinct vertices. Then either $G$ has 
 five connected disjoint subgraphs $X_1, X_2, \dots, X_5$ such that $X_i$ contains $v_i$ for every $i = 1,2,3$
 and for every $j = 4,5$ $X_j$ has neighbour in each $X_i$ for all $i = 1,2,3$ or $G$ is planar such that $v_1, v_2, v_3$ are on boundary.
\end{thm}

\begin{thm}
 Let $G$ be a 4-connected graph and $v_1, v_2, v_3, v_4$ be four distinct vertices. Then either $G$ has $K_4$ minor rooted
 at $\{v_1, v_2, v_3, v_4\}$ or $G$ is planar such that $v_1, v_2, v_3, v_4$ are on the boundary.
\end{thm}

Those two results were the starting points for rooted minor problems. 
\newline
It turns out that rooted minors are not only
useful for the proof of graph Minor theorem, but also for some structure theorems which are used to prove some existence
of graph minor, some of which are presented below:

Robertson, Seymour, and Thomas \cite{robertson_seymour_1993} used rooted $K_4$–minors to prove the case $k=5$ of Hadwiger's conjecture.
Kawarabayashi and Toft \cite{Kawarabayashi2005} used rooted minors to prove that every 7-chromatic graph has $K_7$ or $K_{4,4}$ as minor.

In graphs embedded on surfaces, the concept of a rooted minor extends naturally to problems involving
face covers. A recent paper \cite{Fiorini2025} shows that in a 3-connected
graph embedded in a surface of Euler genus $g$, if the graph has no rooted $K_{2,t}$ minor,
then there exists a face cover whose size is bounded by a function of $g$ and $t$. 
In the planar case, they got $O(t^4)$ upper bound, which improved the result of Böhme and Mohar~\cite{BOHME2002291}.